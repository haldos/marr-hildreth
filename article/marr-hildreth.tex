%-------------------------------------------------------------------------------
% Edge detectors review
% by Haldo Spontón & Juan Cardelino
% IPOL 2012
%-------------------------------------------------------------------------------
\documentclass{ipol}
\ipolSetYear{2012}
\ipolSetDOI{10.XXXX/ipol.2012.sc-mhed}

\usepackage{hyperref,verbatim,graphicx,amsmath,amssymb,amssymb,dsfont}
\usepackage[ruled,linesnumbered]{algorithm2e}
\newtheorem{theorem}{Theorem}
\begin{document}

%-------------------------------------------------------------------------------
\title{Marr-Hildreth Edge Detector}
\author{Haldo Spont\'on,\\
        Juan Cardelino}
\date{}
\ipolMaketitle

%-------------------------------------------------------------------------------
\begin{ipolAbstract}
In this work...
%The Marr-Hildreth algorithm is a method of detecting edges in digital 
%images. It's based on convolving the image with the Laplacian of the 
%Gaussian function. Then, zero crossings are detected in the filtered 
%result to obtain the edges points.
\end{ipolAbstract}

%-------------------------------------------------------------------------------
\begin{ipolCode}
The C implementation of the Marr-Hildreth algorithm (beta version) can be
downloaded \href{http://iie.fing.edu.uy/~haldos/ipol/marr-hildreth.tar.gz}{here}.
\end{ipolCode}

%-------------------------------------------------------------------------------
%\begin{ipolSupp}
%\end{ipolSupp}

%-------------------------------------------------------------------------------
\section{Introduction}
\label{sec:intro}

In the 70's, edge detection methods were based on using small operators 
(such as Sobel masks), attempting to approximate the computation of the
first derivative of the image. In section ~\ref{sec:sobel} we study the 
performance of these algorithms, as an introduction to a more sophisticated 
analysis into the edge finding process.\\

In 1980 Marr and Hildreth argued that intensity changes are not independent 
of image scale, so the edge detection requires the use of different size 
operators. They also argued that a sudden intensity change will be seen 
as a peak (or trough) in the first derivative or, equivalently, as a zero 
crossing in the second derivative. The bulk of this work focuses on the 
study of this algorithm, presented in Section ~\ref{sec:mh}.\\

In section ~\ref{sec:examples} we present some examples to compare the 
performance of the algorithms studied in this work. We also present a 
video created using the Marr-Hildreth algorithm, applied frame by frame.\\

%-------------------------------------------------------------------------------
\section{First approach: Sobel, Prewitt \& Roberts}
\label{sec:sobel}

%-------------------------------------------------------------------------------
\section{Marr-Hildreth Algorithm}
\label{sec:mh}

The Marr-Hildreth algorithm is a method of detecting edges in digital 
images. It's based on find zero crossing points of the second derivative
of the image. This can be done in several ways. In this work we implemented 
two different ways; convolving the image with a Gaussian kernel and then 
approximating the second derivative (Laplacian) with a 3x3 kernel, or 
convolving the image with a kernel calculated as the Laplacian of a 
Gaussian function. There are more ways to do so, for example, using 
recursive Gaussian filters.\\

%convolving the image with the Laplacian of the 
%Gaussian function. Then, zero crossings are detected in the filtered 
%result to obtain the edges points.\\

The algorithm is divided in three steps, each one described later:
\begin{enumerate}
	\item Grayscale conversion of the input image.
	\item Convolution of the image with:
	\begin{itemize}
		\item a Laplacian of Gaussian kernel. (or)
		\item a Gaussian kernel and then a Laplacian operator.
	\end{itemize}
	\item Search of zero crossing points in the filtered image.\\
\end{enumerate}

Some auxiliary functions were implemented for computing the operations
needed in the algorithm, such as Gaussian kernel and Laplacian of a Gaussian 
kernel generation, and 2-D convolution of an image with a given kernel, 
with zero padding.\\

%-------------------------------------------------------------------------------
\subsection{Grayscale conversion}

The first step is to convert color images to gray intensity images. We use
the libpng coefficients to do this:
$$
Y = (6968 R + 23434 G + 2366 B) / 32768
$$
This is a basic operation but has to be clearly defined.\\

%-------------------------------------------------------------------------------
\subsection{Convolution with kernel}

We need to smooth the image using a Gaussian kernel, and then compute numerically the second derivative of the result. As we said before, there are two ways of doing this:\\
\begin{itemize}
	\item Convolve the image with a Gaussian kernel and then compute numerically the second derivative of the smoothed image.
	\item Calculate analytically the second derivative of the Gaussian function, and convolve the image with that result function (Laplacian of the Gaussian).\\
\end{itemize}

%-------------------------------------------------------------------------------
\subsection{Zero crossing}

%-------------------------------------------------------------------------------
\section{Examples}
\label{sec:examples}

%-------------------------------------------------------------------------------
\subsection{Video}

This is an example of applying the Marr-Hildreth edge detector, frame by frame, to a video:
\begin{itemize}
	\item \href{http://iie.fing.edu.uy/~haldos/ipol/video.mov}{original}(xxxMb).
	\item \href{http://iie.fing.edu.uy/~haldos/ipol/video.marr-hildreth.mp4}{marr-hildreth version}(xxxMb).
\end{itemize}

%-------------------------------------------------------------------------------
\begin{thebibliography}{9}

\end{thebibliography}

\end{document}
%-------------------------------------------------------------------------------
